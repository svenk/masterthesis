%& -aux-directory=/tmp
% sorgt  dafuer dass aux files sonstwohin kommen -output-directory=C:/pdfout
\documentclass[10pt,a4paper, fleqn]{article}
% twocol class oder so geht auch
% fleqn macht align formeln nach lings
\usepackage[utf8]{inputenc}
\usepackage{hyperref}
\hypersetup{linktocpage}
%\usepackage[ngerman]{babel}
\usepackage[american]{babel}
\usepackage{amsmath} % xrightarrow, ...
\usepackage{cite}
\usepackage{units} % nicefrac
\usepackage{datetime} % fuer Uhrzeit im \date
%\usepackage{wrapfig} % bilder rechts
\usepackage{caption} % fuer subcaption
\usepackage{subcaption} % subfigures
\usepackage{graphicx} % Bilder allgemein einbinden
%\usepackage{tabularx} % Tabellen
\usepackage{lastpage} % Anzahl Seiten
\usepackage{multicol} % zweispaltige Titelseite
\usepackage{a4wide} % bessere Papiernutzung
\usepackage{fancyhdr} % Header/Footer
%\pagestyle{fancy} % Kopf/Fussbereich der Seiten
\usepackage{amssymb} % therefore = dreieckdots
\usepackage{array} % tables
\usepackage{booktabs} % better tables
\usepackage{floatrow} % caption beside image
\usepackage[toc,page]{appendix}
\usepackage{mathtools} % ceil, floor
\DeclarePairedDelimiter\ceil{\lceil}{\rceil}
\DeclarePairedDelimiter\floor{\lfloor}{\rfloor}

\usepackage[usenames,dvipsnames]{color} % more colors

% zweispaltiger Text
\usepackage{multicol}
%\setlength{\columnseprule}{0.4pt}

% Ueberschriften kleiner 	
%\usepackage{titlesec}
%\titleformat{\section}{\large\bfseries}{\thesection}{1em}{}
%\titlespacing{\paragraph}{%
%  0pt}{%              left margin
%  0.5\baselineskip}{% space before (vertical)
%  1em}%               space after (horizontal)
%%\titlespacing{\section}{0pt}{0.2\baselineskip}{0.1\baselineskip}
%\titlespacing{\align}{0pt}{0.2\baselineskip}{0.1\baselineskip}
%\titlespacing{\equation}{0pt}{0.2\baselineskip}{0.1\baselineskip}

% abgefahrenes highlighting von formeln
\usepackage{xcolor}
% klappt net, was einfacheres:
\newcommand{\highlight}[1]{%
  \colorbox{green!30}{$\displaystyle#1$}}
  
% Hübsche Boxen
% Alternative dazu (gut fuer Seitenanmerkungen):
% http://tex.stackexchange.com/a/73418/49958
% \usepackage[draft]{todonotes}   % notes showed
\usepackage{mdframed}
\mdfdefinestyle{todo}{
	linecolor=red,
	backgroundcolor=yellow!40,
	linewidth=5pt,
	topline=false,
	rightline=false,
	bottomline=false
}
% definiere \begin{todo} und \todo{blabla} auf einmal:
\usepackage{environ}
\NewEnviron{todo}%[1]
  {\begin{mdframed}[style=todo]
  	 \BODY
   \end{mdframed}\vspace{.3cm}}

% Kopfzeile/Fusszeile mit fancy
%\fancyhead{}
%\fancyfoot{}
%\fancyfoot[FL]{\slshape F-Praktikum, Supraleiter}
%\fancyfoot[FR]{\slshape Page \thepage {} / \pageref*{LastPage}}
%\renewcommand{\headrulewidth}{0 pt}

% Bibliography
\bibliographystyle{ieeetr}

% Farben (werden derzeit nur in hypersetup verwendet)
\usepackage{color}
\definecolor{darkblue}{rgb}{0,0,.6}
\definecolor{darkred}{rgb}{.1,0,0}
\definecolor{darkgreen}{rgb}{0,.5,0}

% Schriften
% Palatino for rm and math | Helvetica for ss | Courier for tt
\usepackage{mathpazo} % math & rm
\linespread{1.05}        % Palatino needs more leading (space between lines)
\usepackage[scaled]{helvet} % ss
\usepackage{courier} % tt
\normalfont
\usepackage[T1]{fontenc}

% Hyperref aufsetzen
\hypersetup{
    pdftitle={Master Physik bei Nicolini, Calc writeup},
    pdfauthor={Sven Köppel},
    pdfsubject={master},
    pdfkeywords={physik} {master} {uni} {frankfurt} {fias},
    colorlinks=true,        % test: stat gerahmten Links
    linkcolor=red,          % color of internal links
    citecolor=darkgreen,    % color of links to bibliography
    filecolor=darkred,      % color of file links
    urlcolor=cyan           % color of external links
}

% Allgemeine Meta-Daten, derzeit ungenutzt
\title{\vspace{-9ex} Calc15 \vspace{-1ex}} % vertikalen platz weg...
\author{\small %
\href{https://itp.uni-frankfurt.de/~koeppel}{Sven Köppel} \\
\small \texttt{koeppel@fias.uni-frankfurt.de}}
\date{\small Generation date: \today, \currenttime}


\begin{document}
\maketitle

% abkuerzungen:
\renewcommand{\d}{\mathrm{d}}
\newcommand{\dd}[2]{\frac{\mathrm{d} #1}{\mathrm{d} #2}}
\newcommand{\pp}[2]{\frac{\partial #1}{\partial #2}}
\renewcommand{\L}{L_P}
\newcommand{\pr}{p_r}
\newcommand{\psenk}{p_\perp}
\newcommand{\ebenso}{\biggl( ~ \therefore ~ \biggr) }
\newcommand{\metrik}[1]{\d s^2 = \left( #1 \right) \d t^2 \left( #1 \right)^{-1} \d r^2 + r^2 \d \Omega_{D-2}^2 }
\newcommand{\winkel}{r^2 \d \Omega^2}
\newcommand{\dann}{$\rightarrow~$}
\newcommand{\CA}{ {\cal A}}
\newcommand{\C}[1]{ {\cal #1}}
\newcommand{\mn}{_{\mu\nu}}

% colored symbols:
% http://tex.stackexchange.com/questions/85033/colored-symbols
\newcommand*{\mathcolor}{}
\def\mathcolor#1#{\mathcoloraux{#1}}
\newcommand*{\mathcoloraux}[3]{%
  \protect\leavevmode
  \begingroup
    \color#1{#2}#3%
  \endgroup
}
% In Text: $a\textcolor{red}{\ast}b$
% In Math: $a\mathcolor{red}{\ast}b$
\newcommand{\redmin}{\mathcolor{red}{-}}
\newcommand{\redplus}{\mathcolor{red}{+}}
\newcommand{\pn}{\mathcolor{OliveGreen}{+ n}}
\newcommand{\n}{ {\mathcolor{OliveGreen}{n}} }

\begin{multicols}{2}
\begin{abstract}
This document effective Quantum Gravity approaches
investigated in the Calc series so far in respect
to their divergence curing behaviour.

This is done for the Journal Club presentation at
18. June 2014.
\end{abstract}
\vfill
\columnbreak
\tableofcontents
\end{multicols}

\section{Divergence of General Relativity}
GR exhibits an ultra-violet divergence, that is, expressions
typically diverge when $r\to 0 \Leftrightarrow p \to~\infty$
in spherical symmetry. Consider the spatial flat integration measure in four Spacetime dimensions:
\begin{equation} \label{eq:Intro1}
\int \d^3 p \propto \int \limits_{-\infty}^{\infty} p^2~\d p
= \lim_{\Lambda \to \infty} \int \limits_{-\Lambda}^\Lambda p^2~\d p
\propto \Lambda^3
\end{equation}
When examining self-regular black hole solutions, we examine expressions which cure this divergencies. They typically look like
\begin{equation}
\int \frac{ \d^3 p }{ f(p) }
\end{equation}
with polynomial functions $f(p)$ that manages a »soft cutoff«. For example, in the GUP principle [Kempf2005], it is $f(p) = 1 + \beta p^2$. The series expansion of $1/f(p)$ at $p\to\infty$ (which corresponds with $\beta \to 0$) is
\begin{equation} \label{eq:GUPexp}
\frac{1}{1 + \beta p^2} \approx \frac{1}{\beta p^2} 
- \C O\left( \frac{1}{\beta^2 p^4} \right)
% + \frac{1}{\beta^3 p^6}
\end{equation}
Therefore, we can understand the integration modification as
\begin{equation} \label{eq:Mod1}
\int \frac{ \d^3 p }{ 1 + \beta p^2}
\propto
\int \limits_{-\infty}^{\infty} \frac{ p^2~\d p }{ 1 + \beta p^2}
\stackrel{\text{\eqref{eq:GUPexp}}}{\approx}
\int \limits_{-\infty}^{\infty} \frac{\d p}{\beta}
\end{equation}
This is good. We like that.

\subsection{What $f(p)$ has to archive in higher dimensions}
It is obvious that $f(p)$ must scale with the number of extra dimensions, because \eqref{eq:Intro1} gets
\begin{equation}
\int \d^{3+n} p
\approx \int \limits_{-\infty}^\infty
p^{2+n}~\d p
\end{equation}
Thus the most simple extension of Kempf would be $f(p) = 1 + L^{2+n} p^{2+n}$, with $\beta = L^2$ and $L$ the reduced higher dimensional Planck length. It is easy to show that, using this approach, \eqref{eq:Mod1} again gets $\propto \int \d p$.

\subsection{How $f(p)$ is archieved with my $H$-models}
This section ties on the formalism I introduced in Calc14 -- which is merely the name »$H$-model« for the approach of talking about the holographic metric ($h(r)$ profile), self-encoding metric ($h_\alpha(r)$ profile) and eventually the Bardeen metric ($h_e(r)$ profile).

In my work, a fourier transformation is typically introduced like
\begin{equation}
\mathcolor{gray}{\C A^{-2}(p^2) =}
\int \d^{3+n} r~\left( \frac{1}{r^{n+2}} \dd{H(r)}r \right)
\mathcolor{gray}{~e^{-ipr}}
\end{equation}
The factor $r^{n+2}$ in the denominator is placed there »by design«, as all $H$-models have a matter density
\begin{equation}
\rho(r) = \frac{M}{\Omega_{n+2} r^{n+2}} H'(r)
\end{equation}
with the $(n+2)$-surface (spatial surface) in the denominator. When inserting $H(r) = \Theta(r)$, $H'(r) = \delta(r)$, one ends up in the Schwarzschild(-Tangherlini) case. That is, everything is fine in ordinary Schwarzschild:
\begin{equation} \label{eq:Tangherlini1}
\int \d^{3+n} r \left( \frac{1}{r^{n+2}} \delta(r) \right)
\propto \int \limits_{-\infty}^\infty
\d r~ r^{2+n} \left( \frac{1}{r^{n+2}} \delta(r) \right)
= \int \d r~\delta(r)
\end{equation}
Caution must be made when performing the $(3+n)\to 1$ dimensional integral rewrite, since an alternating $(-1)^n$ inserts the integrand. This technical detail was first found in Calc13 and discussed in Calc14.

So it looks like in my calculations, $H(r)$ does not need to scale with the number of extradimensions $n$. This is really weird, I always thought it has to scale. Hm.





\end{document}
