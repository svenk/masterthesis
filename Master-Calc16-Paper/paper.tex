%& -aux-directory=/tmp
% sorgt  dafuer dass aux files sonstwohin kommen -output-directory=C:/pdfout
\documentclass[10pt,a4paper, fleqn]{article}
% twocol class oder so geht auch
% fleqn macht align formeln nach lings
\usepackage[utf8]{inputenc}
\usepackage{hyperref}
\hypersetup{linktocpage}
%\usepackage[ngerman]{babel}
\usepackage{amsmath} % xrightarrow, ...
\usepackage{cite}
\usepackage{units} % nicefrac
\usepackage{datetime} % fuer Uhrzeit im \date
%\usepackage{wrapfig} % bilder rechts
\usepackage{caption} % fuer subcaption
%\usepackage{subcaption} % subfigures
\usepackage{graphicx} % Bilder allgemein einbinden
%\usepackage{tabularx} % Tabellen
\usepackage{lastpage} % Anzahl Seiten
\usepackage{multicol} % zweispaltige Titelseite
\usepackage{a4wide} % bessere Papiernutzung
\usepackage{fancyhdr} % Header/Footer
%\pagestyle{fancy} % Kopf/Fussbereich der Seiten
\usepackage{amssymb} % therefore = dreieckdots
\usepackage{array} % tables
\usepackage{booktabs} % better tables
\usepackage{floatrow} % caption beside image

% zweispaltiger Text
\usepackage{multicol}
%\setlength{\columnseprule}{0.4pt}

% Ueberschriften kleiner 	
%\usepackage{titlesec}
%\titleformat{\section}{\large\bfseries}{\thesection}{1em}{}
%\titlespacing{\paragraph}{%
%  0pt}{%              left margin
%  0.5\baselineskip}{% space before (vertical)
%  1em}%               space after (horizontal)
%%\titlespacing{\section}{0pt}{0.2\baselineskip}{0.1\baselineskip}
%\titlespacing{\align}{0pt}{0.2\baselineskip}{0.1\baselineskip}
%\titlespacing{\equation}{0pt}{0.2\baselineskip}{0.1\baselineskip}

% abgefahrenes highlighting von formeln
\usepackage{xcolor}
% klappt net, was einfacheres:
\newcommand{\highlight}[1]{%
  \colorbox{green!30}{$\displaystyle#1$}}

% Kopfzeile/Fusszeile mit fancy
%\fancyhead{}
%\fancyfoot{}
%\fancyfoot[FL]{\slshape F-Praktikum, Supraleiter}
%\fancyfoot[FR]{\slshape Page \thepage {} / \pageref*{LastPage}}
%\renewcommand{\headrulewidth}{0 pt}

% Farben (werden derzeit nur in hypersetup verwendet)
\usepackage{color}
\definecolor{darkblue}{rgb}{0,0,.6}
\definecolor{darkred}{rgb}{.1,0,0}
\definecolor{darkgreen}{rgb}{0,.5,0}

% Schriften
% Palatino for rm and math | Helvetica for ss | Courier for tt
\usepackage{mathpazo} % math & rm
\linespread{1.05}        % Palatino needs more leading (space between lines)
\usepackage[scaled]{helvet} % ss
\usepackage{courier} % tt
\normalfont
\usepackage[T1]{fontenc}

% Hyperref aufsetzen
\hypersetup{
    pdftitle={Master Physik bei Nicolini, Calc writeup},
    pdfauthor={Sven Köppel},
    pdfsubject={master},
    pdfkeywords={physik} {master} {uni} {frankfurt} {fias},
    colorlinks=true,        % test: stat gerahmten Links
    linkcolor=red,          % color of internal links
    citecolor=darkgreen,    % color of links to bibliography
    filecolor=darkred,      % color of file links
    urlcolor=cyan           % color of external links
}

% Allgemeine Meta-Daten, derzeit ungenutzt
\title{\vspace{-9ex} Paper \vspace{-1ex}} % vertikalen platz weg..
\author{\small %
\href{https://itp.uni-frankfurt.de/~koeppel}{Sven Köppel} \\
\small \texttt{koeppel@fias.uni-frankfurt.de}}
\date{\small Generation date: \today, \currenttime}


\begin{document}
\maketitle

% abkuerzungen:
\renewcommand{\d}{\mathrm{d}}
\newcommand{\dd}[2]{\frac{\mathrm{d} #1}{\mathrm{d} #2}}
\newcommand{\pp}[2]{\frac{\partial #1}{\partial #2}}
\renewcommand{\L}{L_P}
\newcommand{\pr}{p_r}
\newcommand{\psenk}{p_\perp}
\newcommand{\ebenso}{\biggl( ~ \therefore ~ \biggr) }
\newcommand{\metrik}[1]{\d s^2 = \left( #1 \right) \d t^2 \left( #1 \right)^{-1} \d r^2 + r^2 \d \Omega_{D-2}^2 }
\newcommand{\winkel}{r^2 \d \Omega^2}
\newcommand{\dann}{$\rightarrow~$}
\newcommand{\CA}{ {\cal A}}
\newcommand{\C}[1]{ {\cal #1}}
\newcommand{\mn}{_{\mu\nu}}
\newcommand{\surface}{\Omega_{n+2}}

\newcommand{\lesen}{ {\bf LESEN} }
\newcommand{\Ms}{M_\star}
\newcommand{\Ls}{L_\star}
\newcommand{\Gs}{G_\star}

\begin{multicols}{2}
This is Paper1, coming after Calc12.

{\bf Title:
Holographic screens in ultraviolet
self-complete quantum gravity and
large extra dimensions
}
\vfill

\columnbreak
\tableofcontents
\end{multicols}

\subsection*{Abstract}
In this paper we study the geometry and the thermodynamics of a
{\it holographic screen} in the framework of the ultraviolet
self-complete quantum gravity. Large extra dimensions address
the gauge hierarchy problem and allow the effective fundamental
scale be not far from 1 TeV. In this paper we show that
holographic principles recently presented by
Nicolini and Spalucci can be smoothly extended with flat,
torodially compactified extra dimensions.

\section*{Sven: Reference Guide}
Wesentliche Vorlagen hierfür sind:

\begin{itemize}
\item Rizzo2006 \cite{Rizzo}: NC + LXD
\item N+Spallucci \cite{NS2012, NS2013}: Holographic screens in ultraviolet self-complete quantum gravity stellen jeweils $h$ und $h_\alpha$ vor.
\item Isi, Mureika, N \cite{Isi1,Isi2}: GUP, aber als Rahmenvorlage von N definiert
\item Bleicher, N 2010: LXD at LHC
\item Dickes13: No minimal length neccessary
\end{itemize}

\section{Introduction}
Blablabla warum QG toll ist.

\subsection{Ultraviolet Protection}
Authors of [...] have shown that gravity may be self-complete:

Anfangen mit \cite{Isi1}:
\begin{equation}
r_H = \lambda_C
\end{equation}
Minimal BH, dann das Holy Grail-Bild (Isi Abb 1)

\subsection{The large extra-dimensions scenario}
Ähnlich BN2010, Rizzo2006
%
\begin{equation}
M_{\star}^{(2+k)} = M_P^2 / R^k
\end{equation}

Schwarzschild-Tangherlini-Metrik ansprechen

\section{Self-regular black hole solutions}
We start from the energy density for a point-particle in spherical coordinates as
%
\begin{equation}
\label{eq:rho}
\rho(r) = \frac{M}{\Omega_{n+2}~r^{n+2}} \delta(r)
\end{equation}
%
where $\delta(r)$ is the Dirac delta. The energy distribution
implies a black hole for any value of mass $M$ even for subplanckian
values where one expects just particles.

We can express the Dirac delta distribution as the derivative of
a Heaviside step-function $\Theta$,
\begin{equation}
\delta(r) = \dd {\Theta(r)} r
\end{equation}
%
We modify the energy distribution \eqref{eq:rho} in order to
overcome the ambiguities of the Schwarzschild-Metrik by considering
a ``smooth'' function $h(r)$ in place of the Heaviside step
\begin{equation}
\Theta(r) \to h(r)
\end{equation}
%
The new profile is defined throught $h(r)$ by
\begin{equation}
\rho(r) = \frac M {\Omega_{n+2}~r^{n+2}} \dd {h(r)}r \equiv T_0^0
\end{equation}
%
By means of the conservation equation $\nabla_\mu T^{\mu\nu} = 0$
one can determine the remaining components of the stress tensor \cite{Rizzo}
\begin{equation}
\text{\it hier ausführlich machen}
\end{equation}
One eneds up with the metric
\begin{equation}
\d s^2 = - \left(1 - V(r) \right)~\d t^2
+ \left(1 - V(r) \right)^{-1}~\d r^2 
+ r^{2+n}~\d \Omega_{2+n}\dots
\end{equation}
with
\begin{equation}
V(r) =  \frac{2}{n+2} \frac{M}{\Ms^{n+2}} {\frac{1}{\Omega_{n+2}}} \frac{h(r)}{r^{n+1}}
\end{equation}
Carfeully making back the transition $h\to\Theta$, actually
$h\to 1$ in the $r>0$ regime, one ends up
with Schwarzschild-Thangerlini.

If we set the mass aribitriarly
\begin{equation}
\label{eq:M}
M = \frac{n+2}{2} ~\Omega_{n+2}~ \frac{1}{h(r_H)} \left( \frac{r_H}{\Ls} \right)^{n+1} ~\Ms
\end{equation}

M has the physical meaning of a mass for a spherical
{\it holographic screen} with radius $r_H$.

Eigenschaften angeben: $M=...$, Ereignishorizonte $r_h = ...$.

\section{Thermodynamics, area quantization and mass spectrum}
Temp-Plot. $T_H$, $C$ und ganz wichtig Logarithmische Entropie-Korrektur $S$.

Area Quantization: Nachfragen ob ich das auch machen soll. Ist vermutlich schnell gemacht.

\section{Conclusions}
Kurzum: Nichts neues passiert.

We showed that the idea of holography is compatible
with the concept of extra dimensions in space-time.

Oder

We showed that the holographic principle is invariant
under extension of spacial dimensions.




\section{Self-regular }




\bibliographystyle{hep}
\bibliography{literature}{}
%
%\begin{thebibliography}{99}
%\bibitem[Isi13]{isi1} Isi, Mureika, Nicolini Nov 2013
%
%\bibitem[Dvali]{dvali} Dvali, Gomez, "Self-Completeness of Einstein Gravity"
%
%\bibitem{hossi} Hossenfelder, "minimal Length Scale Scenarios for QG" Living Rev Relativity 16, 2013
%
%\bibitem{spallici11} Spallucci Ansoldi "Regular Black holes in UV self-complete QG" 2011
%
%\bibitem{spallucci} "Spallucci Smailagic  "BG production in self-complete QG" 2012
%
%\bibitem{hayward} "Formation and evaporation of regular BHs" 2006
%
%\bibitem{modesto} "LQBHs" 2006,
%Self-dual BHs, Hossi - also ISI13 Refs[23-25]
%  - \lesen
%
%\end{thebibliography}


\end{document}