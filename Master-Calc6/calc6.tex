%& -aux-directory=/tmp
% sorgt  dafuer dass aux files sonstwohin kommen -output-directory=C:/pdfout
\documentclass[10pt,a4paper, fleqn]{article}
% twocol class oder so geht auch
% fleqn macht align formeln nach lings
\usepackage[utf8]{inputenc}
\usepackage{hyperref}
\hypersetup{linktocpage}
%\usepackage[ngerman]{babel}
\usepackage{amsmath} % xrightarrow, ...
\usepackage{cite}
\usepackage{units} % nicefrac
\usepackage{datetime} % fuer Uhrzeit im \date
%\usepackage{wrapfig} % bilder rechts
\usepackage{caption} % fuer subcaption
%\usepackage{subcaption} % subfigures
%\usepackage{graphicx} % Bilder allgemein einbinden
%\usepackage{tabularx} % Tabellen
\usepackage{lastpage} % Anzahl Seiten
\usepackage{multicol} % zweispaltige Titelseite
\usepackage{a4wide} % bessere Papiernutzung
\usepackage{fancyhdr} % Header/Footer
%\pagestyle{fancy} % Kopf/Fussbereich der Seiten
\usepackage{amssymb} % therefore = dreieckdots
\usepackage{array} % tables

% zweispaltiger Text
\usepackage{multicol}
%\setlength{\columnseprule}{0.4pt}

% Ueberschriften kleiner 	
%\usepackage{titlesec}
%\titleformat{\section}{\large\bfseries}{\thesection}{1em}{}
%\titlespacing{\paragraph}{%
%  0pt}{%              left margin
%  0.5\baselineskip}{% space before (vertical)
%  1em}%               space after (horizontal)
%%\titlespacing{\section}{0pt}{0.2\baselineskip}{0.1\baselineskip}
%\titlespacing{\align}{0pt}{0.2\baselineskip}{0.1\baselineskip}
%\titlespacing{\equation}{0pt}{0.2\baselineskip}{0.1\baselineskip}

% abgefahrenes highlighting von formeln
\usepackage{xcolor}
% klappt net, was einfacheres:
\newcommand{\highlight}[1]{%
  \colorbox{green!30}{$\displaystyle#1$}}

% Kopfzeile/Fusszeile mit fancy
%\fancyhead{}
%\fancyfoot{}
%\fancyfoot[FL]{\slshape F-Praktikum, Supraleiter}
%\fancyfoot[FR]{\slshape Page \thepage {} / \pageref*{LastPage}}
%\renewcommand{\headrulewidth}{0 pt}

% Bibliography
\bibliographystyle{ieeetr}

% Farben (werden derzeit nur in hypersetup verwendet)
\usepackage{color}
\definecolor{darkblue}{rgb}{0,0,.6}
\definecolor{darkred}{rgb}{.1,0,0}
\definecolor{darkgreen}{rgb}{0,.5,0}

% Schriften
% Palatino for rm and math | Helvetica for ss | Courier for tt
\usepackage{mathpazo} % math & rm
\linespread{1.05}        % Palatino needs more leading (space between lines)
\usepackage[scaled]{helvet} % ss
\usepackage{courier} % tt
\normalfont
\usepackage[T1]{fontenc}

% Hyperref aufsetzen
\hypersetup{
    pdftitle={Master Physik bei Nicolini, Calc writeup},
    pdfauthor={Sven Köppel},
    pdfsubject={master},
    pdfkeywords={physik} {master} {uni} {frankfurt} {fias},
    colorlinks=true,        % test: stat gerahmten Links
    linkcolor=red,          % color of internal links
    citecolor=darkgreen,    % color of links to bibliography
    filecolor=darkred,      % color of file links
    urlcolor=cyan           % color of external links
}

% Allgemeine Meta-Daten, derzeit ungenutzt
\title{\vspace{-9ex} Calc6 \vspace{-1ex}} % vertikalen platz weg...
\author{\small %
\href{https://itp.uni-frankfurt.de/~koeppel}{Sven Köppel} \\
\small \texttt{koeppel@fias.uni-frankfurt.de}}
\date{\small Generation date: \today, \currenttime}


\begin{document}
\maketitle

% abkuerzungen:
\renewcommand{\d}{\mathrm{d}}
\newcommand{\dd}[2]{\frac{\mathrm{d} #1}{\mathrm{d} #2}}
\renewcommand{\L}{L_P}
\newcommand{\pr}{p_r}
\newcommand{\psenk}{p_\perp}
\newcommand{\ebenso}{\biggl( ~ \therefore ~ \biggr) }
\newcommand{\metrik}[1]{\d s^2 = \left( #1 \right) \d t^2 \left( #1 \right)^{-1} \d r^2 + r^2 \d \Omega_{D-2}^2 }
\newcommand{\winkel}{r^2 \d \Omega^2}
\newcommand{\dann}{$\rightarrow~$}

\section{The surface issue}
These calculations point out a missing surface term when doing the plausibility check for the holographic approach done in [NS11.2013].

I work in $D=4+n$ dimensions, but any equation must hold for $n=0$, too. I follow the Rizzo2006 deviation (also performed in Calc1) for deriving an ODE for the potential $V(r)$ in $g_{00}=1-V(r)$ SMM like metrics. Having given only $\rho(r)$, the continuity equation $T^{AB}_{;B}=0$ gives
\begin{equation}
T_i^i = \rho + \frac{r}{n+2} \partial_r \rho.
\end{equation}
Which then lead to the first order differential equation
\begin{equation}
V' + \frac{n+1}{r} V = \frac{1}{M_*^{n+2}} \frac{2r\rho}{n+2}. \label{ode}
\end{equation}
The Ansatz $V(r)=r^{-(n+1)}\left( C \int^r x^2 \rho(x) \d x + D\right)$ solves equation \ref{ode}. It is simple to derive, as done in Calc3 and Calc4,
\begin{equation}
V(r) = \frac{1}{r^{n+1}} \left( \frac{2}{(n+2)M_*^{n+2}} \int_{c_1}^r x^{n+2} \rho(x) \d x + c_2 \right). \label{sol}
\end{equation}

It is important to remark that the integral in \ref{sol} only looks like the radial part of an partially performed spherical integration, but {\it there is no surface term}, as there would be if the integral really would be $m(r)=\int \d^{n+3}~ \vec{r} ~\rho(\vec r)$. That is,

\begin{equation}
\int \d^{n+3} \vec{r} ~\rho(\vec{r}) = \int \d r ~\left( \Omega_{n+2} r^{n+2} \right)~ \rho(r), \label{reg}
\end{equation}

With $\Omega_{n+2}r^{n+2}$ being the $(n+2)$ dimensional surface (of an ${n+3}$ dimensionall sphere)

\begin{equation}
%\displaystyle
\Omega_{n+2} = 2 \frac{\pi^\frac{n+3}{2}}{\Gamma\left(\frac{n+3}{2}\right)}
\end{equation}

The missing $\Omega_{n+2}$ in eq. \ref{sol} compared to \ref{reg} stands out. This is important, because the holographic approach depends on that property of \ref{reg}.

\subsection{NC in $D$ dim}
Rizzo introduces the reduced Planck scale $M_*$ by $M_P^2 = V_n M_*^{n+2}$, with $v_n = (2 \pi R_c)^n$ the volume of the compacted dimensions as tori with radius $R_c$. Thus the $n\to 0$ limit gives $M_*^2 = M_P^2 = 1 / G$.

%The properties of the gaussian $\rho(r)$ enable $V(r)$ to kill the surface term $\Omega_{n+2}$ in the $n\to 0$ limit, since it looks like [Rizzo $A(r)$]

Using the gaussian $\rho(r)$, Rizzo (and I in Calc3) got

\begin{equation}
V(r) = \frac{M}{M_*^{n+2}} \frac{1}{(n+2) \pi^{(n+3)/2}} \frac{1}{r} \Gamma\left( \frac{3+n}{2} ; \frac{r^2}{4 \theta} \right)
\end{equation}

In the $\theta,n \to 0$ limit, $\Gamma(\frac{3}{2}; \infty) = \sqrt{\pi}/2$ and therefore we end with

\begin{equation}
V(r) = \frac{GM}{4 \pi r }
\end{equation}

\subsection{$h(r)$ Profile}
In [NS 07.11.2013], the $\theta \to h(r)$ smearing function is introduced, so $\partial_r \theta = \delta \to \partial_r h$ enters a smeared density:

\begin{equation}
\rho(r) = \frac{M}{4 \pi r^2} \dd{h}{r}
\quad \xrightarrow{\text{D=n+2 dimensions}} \quad
\rho(r) = \frac{M}{\Omega_{n+2} r^{n+2}} \dd{h}{r}
\end{equation}

Since $\Omega_{2} = 4\pi$, this seems to be true. I showed already in Calc2 that an integration (like in eq~\ref{reg}) over that class of $\rho(r)$ gets trivial in {\it any} dimension.

Lets apply the solution for $V(r)$ at this density. Since that integration is not a {\it full} one, it allows the surface constant $\Omega_{n+2}$ to enter the metric. We end up with (already showed in Calc4)

\begin{equation}
V(r) = \frac{2}{n+2} \frac{M}{M_*^{n+2}} \highlight{\frac{1}{\Omega_{n+2}}} \frac{h(r)}{r^{n+1}}.
\end{equation}

This equation cannot produce the SMM value $V(r) = \frac{2GM}{r}$ any more, because nothing kills the $\Omega_2=4\pi$. Indeed, if we use the Schwarzschild-Tangherlini density $\rho(r) = M / (\Omega_{n+2}) \delta(r)$ and apply it to eq. \ref{sol}, [Reall-Review Section 3.2]

\begin{equation}
V(r) = \frac{1}{r^{n+1}} \left( \frac{2}{(n+2) M_*^{n+2}} \frac{M}{\Omega_{n+2}} \int \d r \delta(r) \right) = \frac{\mu}{r^{n+2}}, \quad \mu = \frac{16\pi G M}{(n+2) \Omega_{n+2}} \label{ST}
\end{equation}

{\bf TODO: Why $\mu$?}. If we now send $n \to 0$, this does not reproduce SMM at all.

The Planck length $M_P^2 = V_n  M_*^{n+2}$ is equal to $M_*$ in $n=0$ dimensions. Since $M_P = 1/\sqrt{G}$ Newtons constant $G = 1/M_*^2$ is restored. Thus, from eq. \label{ST} we get for $n=0$

\begin{equation}
V(r) = \frac{GM}{r}\frac{1}{4\pi}
\end{equation}

Conclusion: There seems always the factor {$8\pi$} to be missing. There seems to be some $G~\leftrightarrow~8\pi G$ issue.


\end{document}