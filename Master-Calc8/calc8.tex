%& -aux-directory=/tmp
% sorgt  dafuer dass aux files sonstwohin kommen -output-directory=C:/pdfout
\documentclass[10pt,a4paper, fleqn]{article}
% twocol class oder so geht auch
% fleqn macht align formeln nach lings
\usepackage[utf8]{inputenc}
\usepackage{hyperref}
\hypersetup{linktocpage}
%\usepackage[ngerman]{babel}
\usepackage{amsmath} % xrightarrow, ...
\usepackage{cite}
\usepackage{units} % nicefrac
\usepackage{datetime} % fuer Uhrzeit im \date
%\usepackage{wrapfig} % bilder rechts
\usepackage{caption} % fuer subcaption
%\usepackage{subcaption} % subfigures
\usepackage{graphicx} % Bilder allgemein einbinden
%\usepackage{tabularx} % Tabellen
\usepackage{lastpage} % Anzahl Seiten
\usepackage{multicol} % zweispaltige Titelseite
\usepackage{a4wide} % bessere Papiernutzung
\usepackage{fancyhdr} % Header/Footer
%\pagestyle{fancy} % Kopf/Fussbereich der Seiten
\usepackage{amssymb} % therefore = dreieckdots
\usepackage{array} % tables

% zweispaltiger Text
\usepackage{multicol}
%\setlength{\columnseprule}{0.4pt}

% Ueberschriften kleiner 	
%\usepackage{titlesec}
%\titleformat{\section}{\large\bfseries}{\thesection}{1em}{}
%\titlespacing{\paragraph}{%
%  0pt}{%              left margin
%  0.5\baselineskip}{% space before (vertical)
%  1em}%               space after (horizontal)
%%\titlespacing{\section}{0pt}{0.2\baselineskip}{0.1\baselineskip}
%\titlespacing{\align}{0pt}{0.2\baselineskip}{0.1\baselineskip}
%\titlespacing{\equation}{0pt}{0.2\baselineskip}{0.1\baselineskip}

% abgefahrenes highlighting von formeln
\usepackage{xcolor}
% klappt net, was einfacheres:
\newcommand{\highlight}[1]{%
  \colorbox{green!30}{$\displaystyle#1$}}

% Kopfzeile/Fusszeile mit fancy
%\fancyhead{}
%\fancyfoot{}
%\fancyfoot[FL]{\slshape F-Praktikum, Supraleiter}
%\fancyfoot[FR]{\slshape Page \thepage {} / \pageref*{LastPage}}
%\renewcommand{\headrulewidth}{0 pt}

% Bibliography
\bibliographystyle{ieeetr}

% Farben (werden derzeit nur in hypersetup verwendet)
\usepackage{color}
\definecolor{darkblue}{rgb}{0,0,.6}
\definecolor{darkred}{rgb}{.1,0,0}
\definecolor{darkgreen}{rgb}{0,.5,0}

% Schriften
% Palatino for rm and math | Helvetica for ss | Courier for tt
\usepackage{mathpazo} % math & rm
\linespread{1.05}        % Palatino needs more leading (space between lines)
\usepackage[scaled]{helvet} % ss
\usepackage{courier} % tt
\normalfont
\usepackage[T1]{fontenc}

% Hyperref aufsetzen
\hypersetup{
    pdftitle={Master Physik bei Nicolini, Calc writeup},
    pdfauthor={Sven Köppel},
    pdfsubject={master},
    pdfkeywords={physik} {master} {uni} {frankfurt} {fias},
    colorlinks=true,        % test: stat gerahmten Links
    linkcolor=red,          % color of internal links
    citecolor=darkgreen,    % color of links to bibliography
    filecolor=darkred,      % color of file links
    urlcolor=cyan           % color of external links
}

% Allgemeine Meta-Daten, derzeit ungenutzt
\title{\vspace{-9ex} Calc8 \vspace{-1ex}} % vertikalen platz weg...
\author{\small %
\href{https://itp.uni-frankfurt.de/~koeppel}{Sven Köppel} \\
\small \texttt{koeppel@fias.uni-frankfurt.de}}
\date{\small Generation date: \today, \currenttime}


\begin{document}
\maketitle

% abkuerzungen:
\renewcommand{\d}{\mathrm{d}}
\newcommand{\dd}[2]{\frac{\mathrm{d} #1}{\mathrm{d} #2}}
\renewcommand{\L}{L_P}
\newcommand{\pr}{p_r}
\newcommand{\psenk}{p_\perp}
\newcommand{\ebenso}{\biggl( ~ \therefore ~ \biggr) }
\newcommand{\metrik}[1]{\d s^2 = \left( #1 \right) \d t^2 \left( #1 \right)^{-1} \d r^2 + r^2 \d \Omega_{D-2}^2 }
\newcommand{\winkel}{r^2 \d \Omega^2}
\newcommand{\dann}{$\rightarrow~$}
\newcommand{\T}{\mathcal{T}}
\newcommand{\A}{\mathcal{A}}

\section{Propagator and modified Einstein Equations}

\subsection{Einstein modifications with NCBHs}
I got plenty of papers following basically modifications $G\to\mathcal{G}$ or $T_{\mu\nu}\to\mathcal{S}_{\mu\nu}$ or $R\to\mathcal{R}$, e.g.

\begin{equation}
R_{\mu \nu} - \frac{1}{2} g_{\mu \nu} R = 8 \pi G_N {\mathcal S}_{\mu \nu}
\quad
{\mathcal S}_{\mu \nu} = {\mathcal F}^2(\square(x)/\Lambda_G^2) T_{\mu\nu}
\quad \text{[MMN 2010]}
\end{equation}

See also

\begin{itemize}
\item {} [N Feb 2012]: {\it Nonlocal and generalized uncertainty principle black holes} \\
Introduces the bi-local distribution $\mathcal{A}^2(x-y)$ that modifies $R$, the Einstein Equations, $T_{\mu\nu}$ and $\delta(\vec x)$
\item{} [Modesto Moffat N Dec 2010]: {\it Black holes in an ultraviolet complete quantum gravity} \\
Introduces the entire (=holomorphic) function $\mathcal{F}$ that scales like $\mathcal{A}$ before.
\item {} [Isi Nov2013]: {\it Self-Completeness and the Generalized Uncertainty Principle} \\
Follows straightforward the [N Feb 2012] formalism.
\item {} [Isi Feb2014]: {\it Self-Completeness in Alternative Theories of Gracity} \\

\end{itemize}

\subsection{A Roadmap}
With the approach $\T_{00} \propto M~ \A^{-2}(\square) \delta(\vec x)$, it's all about finding a differential operator that modifies the Dirac Delta to the smeared functions $\partial_r h$ or $\partial_r h_\alpha$. Since our approach always was $T_{00} \propto M/\Omega ~\delta(r) \to M/\Omega \dd{h}{r}$, with

\begin{align}
h(r) &= \frac{r^2}{r^2 + L^2} \label{eq:holo} \\
h'(r) &= \frac{2rL^2}{(r^2 + L^2)^2} \\
h_\alpha(r) &= \frac{r^{3+n}}{(r^\alpha + L^\alpha/2)^{\frac{3+n}{\alpha}}} \label{eq:selfreg} \\
h'_\alpha(r) &= \frac{(n+3) L^{\alpha } r^{n+2}
   \left(\frac{L^{\alpha
   }}{2}+r^{\alpha
   }\right)^{-\frac{n+3}{\alpha
   }}}{L^{\alpha }+2 r^{\alpha
   }} \label{eq:ultimate}
\end{align}

the propagator $\A^{-2}(\square)$ really must be a complex one to get $\A^{-2} \delta \to h'_\alpha$ (eq. \ref{eq:ultimate})!



\subsection{Note: Spherical Fourier transformation in $3+n$ dimensions}
From Felix Karbstein: Performing the Fourier transform of a generic position space potential $V(|\vec r|)$ in $d=3$ dimensions to momentum space, we obtain

\begin{align}
\hat V(p) &= \int \d^3 r ~e^{-i\vec r \dot \vec p}~ V(r) = 2\pi \int_{-1}^{+1} \d \cos \theta \int_0^\infty \d r ~r^2 ~e^{-i r p \cos \theta} V(r) \\
&= \frac{2 \pi i}{p} \int_0^\infty \d r ~r ~V(r) \left( e^{-irp} - e^{+irp} \right) = \frac{2\pi i}{p} \int_{-\infty}^\infty \d r ~e^{-irp} ~r ~\left[ V(r) \Theta(r) + V(-r) \Theta(-r) \right]
\end{align}

with $r=|\vec r|$, $p=|\vec p|$. Note that this effectively amounts to an one dimensional Fourier transform

\begin{equation}
\hat v(p) = \int_{-\infty}^\infty \d r ~e^{-irp} ~v(r)
\end{equation}

with

\begin{equation}
v(r) = r~\left[ V(r) \Theta(r) + V(-r)\Theta(-r)\right]
\quad\text{and}\quad
\hat V(p) = \frac{2\pi i}p \hat v(p)
\end{equation}

Now proceed with going from $\d^3 r \to \d^{3+n} r$



\end{document}