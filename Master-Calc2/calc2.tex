\documentclass[10pt,a4paper, fleqn]{article}
% twocol class oder so geht auch
% fleqn macht align formeln nach lings
\usepackage[utf8]{inputenc}
\usepackage{hyperref}
\hypersetup{linktocpage}
%\usepackage[ngerman]{babel}
\usepackage{amsmath} % xrightarrow, ...
\usepackage{cite}
\usepackage{units} % nicefrac
\usepackage{datetime} % fuer Uhrzeit im \date
%\usepackage{wrapfig} % bilder rechts
\usepackage{caption} % fuer subcaption
%\usepackage{subcaption} % subfigures
%\usepackage{graphicx} % Bilder allgemein einbinden
%\usepackage{tabularx} % Tabellen
\usepackage{lastpage} % Anzahl Seiten
\usepackage{multicol} % zweispaltige Titelseite
\usepackage{a4wide} % bessere Papiernutzung
\usepackage{fancyhdr} % Header/Footer
%\pagestyle{fancy} % Kopf/Fussbereich der Seiten
\usepackage{amssymb} % therefore = dreieckdots

% zweispaltiger Text
\usepackage{multicol}
%\setlength{\columnseprule}{0.4pt}

% Ueberschriften kleiner
%\usepackage{titlesec}
%\titleformat{\section}{\large\bfseries}{\thesection}{1em}{}
%\titlespacing{\paragraph}{%
%  0pt}{%              left margin
%  0.5\baselineskip}{% space before (vertical)
%  1em}%               space after (horizontal)
%%\titlespacing{\section}{0pt}{0.2\baselineskip}{0.1\baselineskip}
%\titlespacing{\align}{0pt}{0.2\baselineskip}{0.1\baselineskip}
%\titlespacing{\equation}{0pt}{0.2\baselineskip}{0.1\baselineskip}

% abgefahrenes highlighting von formeln
\usepackage{xcolor}
\newlength\dlf
\newcommand\alignedbox[2]{
  % #1 = before alignment
  % #2 = after alignment
  &
  \begingroup
  \settowidth\dlf{$\displaystyle #1$}
  \addtolength\dlf{\fboxsep+\fboxrule}
  \hspace{-\dlf}
  \fcolorbox{red}{yellow}{$\displaystyle #1 #2$}
  \endgroup
}

% klappt net, was einfacheres:
\newcommand{\highlight}[1]{%
  \colorbox{green!30}{$\displaystyle#1$}}

% Kopfzeile/Fusszeile mit fancy
%\fancyhead{}
%\fancyfoot{}
%\fancyfoot[FL]{\slshape F-Praktikum, Supraleiter}
%\fancyfoot[FR]{\slshape Page \thepage {} / \pageref*{LastPage}}
%\renewcommand{\headrulewidth}{0 pt}

% Bibliography
\bibliographystyle{ieeetr}

% Farben (werden derzeit nur in hypersetup verwendet)
\usepackage{color}
\definecolor{darkblue}{rgb}{0,0,.6}
\definecolor{darkred}{rgb}{.1,0,0}
\definecolor{darkgreen}{rgb}{0,.5,0}

% Schriften
% Palatino for rm and math | Helvetica for ss | Courier for tt
\usepackage{mathpazo} % math & rm
\linespread{1.05}        % Palatino needs more leading (space between lines)
\usepackage[scaled]{helvet} % ss
\usepackage{courier} % tt
\normalfont
\usepackage[T1]{fontenc}

% Hyperref aufsetzen
\hypersetup{
    pdftitle={Master Physik bei Nicolini, Calc writeup},
    pdfauthor={Sven Köppel},
    pdfsubject={master},
    pdfkeywords={physik} {master} {uni} {frankfurt} {fias},
    colorlinks=true,        % test: stat gerahmten Links
    linkcolor=red,          % color of internal links
    citecolor=darkgreen,    % color of links to bibliography
    filecolor=darkred,      % color of file links
    urlcolor=cyan           % color of external links
}

% Allgemeine Meta-Daten, derzeit ungenutzt
\title{\vspace{-9ex} Calc2 \vspace{-1ex}} % vertikalen platz weg...
\author{\small %
\href{https://itp.uni-frankfurt.de/~koeppel}{Sven Köppel} \\
\small \texttt{koeppel@fias.uni-frankfurt.de}}
\date{\small Generation date: \today, \currenttime}


\begin{document}
\maketitle

% abkuerzungen:
\renewcommand{\d}{\mathrm{d}}
\newcommand{\dd}[2]{\frac{\mathrm{d} #1}{\mathrm{d} #2}}
\renewcommand{\L}{L_P}
\newcommand{\pr}{p_r}
\newcommand{\psenk}{p_\perp}
\newcommand{\ebenso}{\biggl( ~ \therefore ~ \biggr) }
\newcommand{\metrik}[1]{\d s^2 = \left( #1 \right) \d t^2 \left( #1 \right)^{-1} \d r^2 + r^2 \d \Omega^2 }
\newcommand{\winkel}{r^2 \d \Omega^2}
\newcommand{\dann}{$\rightarrow~$}

\begin{multicols}{2}
Calc2 is the second writeup of notices in my Master thesis
research. This document lists up some formulas and expands some
with that higher dimensional things. 

\vspace{1.3cm}
Definition of frequently used formulas:
%
\begin{align*}
\d s^2 &= -f(r) \d t^2 + f^{-1}(r) \d r^2 + r^2 \d \Omega^2 \\
T_H &= \left( \frac{1}{4 \pi} \dd{g_{00}}{r} \right)_{r=r_H} \\
\gamma(s;x) &= \int_0^x t^{s-1} e^{-t} \d t
\end{align*}
%
\columnbreak
\tableofcontents
\end{multicols}
\section{Summary of papers}
%\rule{0.9\textwidth}{0.4pt}
\begin{multicols}{2}
\subsection{NSS 2006} \label{NSS 2006}
\begin{description}
  \item[Title] Noncommutative geometry inspired Schwarzschild black hole
  \item[Keywords] NC
  \item[Genutzt von] Rizzo 2006
\end{description}

Auch eines der wichtigsten, die ich gelesen habe.


\vfill
\columnbreak
\begin{align}
\highlight{ \rho_\theta(r)}  &= \frac{M}{(4 \pi \theta)^{3/2}} e^{-r^2 / 4 \theta} \\
f(r) &= 1 - \frac{4 M }{r \sqrt{\pi}} \gamma(3/2, r^2/4\theta) \\
r_H & = \frac{4M}{\sqrt{\pi}} \gamma(3/2; r_H^2 / 4\theta) \\
T_H &= \frac{1}{4 \pi r_H} \left( 1 - \frac{r^3_H}{4 \theta^{3/2}} \frac{e^{-r^2_H / 4 \theta}}{\gamma(3/2, r_H^2 / 4 \theta)} \right)  %\nonumber
\end{align}
\vfill

\end{multicols}
\rule{0.9\textwidth}{0.4pt}
\begin{multicols}{2}
\subsection{N Aug 2010}
\begin{description}
  \item[Title] Entropic force, noncommutative gravity and ungravity
  \item[Keywords] Emergent gravity, Verlinde
  \item[Basic Ideas] Newton $F(r)$ herleiten aus $S=k_B \ln N$. Später mit $n$ Raumdimensionen und in $\cal G$ einige Effekte.
\end{description}


\vfill
\columnbreak
\begin{align}
f(r) &= 1 - \frac{2M}{r^{n-2} c^2} {\cal G}(r) \\
F &= \frac{G M m}{r^2} \left( 1 + 4 L^2 \frac{\partial S}{\partial A} \right)
\end{align}

Nicht so passend zum Thema $f(r)$.

\end{multicols}
\rule{0.9\textwidth}{0.4pt}
\pagebreak
\begin{multicols}{2}

\subsection{N Feb 2012}
\begin{description}
  \item[Title] Nonlocal and generalized uncertainity principle black holes
  \item[Keywords] EH-Action
  \item[Basic Ideas] Operator ${\cal A}(x-y)$, running ${\cal G}(r)$, Length scale $l$ of theory
\end{description}

Nicht passend zum Thema.

\vfill
\columnbreak

\begin{align*}
S &= \frac{1}{16 \pi G} \int \d^4 x \sqrt{-g} {\cal R}(x) \\
{\cal R}(x) &= \int \d^4 y \sqrt{-g} {\cal A}^2(x-y) R(y) \\
{\cal A}^2(x-y) &= {\cal A}^2 (\square_x) \delta^4 (x-y) \\
\square_x &= l^2 g_{\mu\nu} \nabla^\mu \nabla^\nu \\
{\cal A}(p^2) &= \exp( {l}^2 p^2 / 2) \dots \\
{\cal T}_{\mu \nu} &= {\cal A}^{-2}(\square) T_{\mu \nu} \\
f(r) &= 1 - \frac{G M \gamma(2;r/\sqrt{\beta})}{r}
\end{align*}

\end{multicols}
\rule{0.9\textwidth}{0.4pt}
\begin{multicols}{2}
\subsection{NS Okt 2012} \label{NS2012}
\begin{description}
  \item[Title] Holographic screens in ultraviolet self-complete quantum gravity
  \item[Keywords] Holography
\end{description}

Das Hauptpaper, was ich als erstes las, darüber geht auch Calc1.

Das Paper umfasst zwei Ansätze, $h_\alpha(r)$ und $h(r)$.

Im ersten Ansatz setzt Bedingung $M_P=M_0$, $M_0=M(r_0)$ das $\alpha = \alpha_0$, $r_0 = L_P$. Im zweiten Ansatz wird eine der drei Bedigungen an eine Metrik verworfen.

\vfill
\columnbreak

\begin{align}
\highlight{ f(r) } &= 1 - \frac{2 M G}{r} h_{\alpha, \dots}(r) \\
\highlight{ h_\alpha(r) } &= \frac{r^3}{(r^\alpha + (\tilde{r}_0)^\alpha / 2)^{3/\alpha}}\\
\highlight{ h(r) } &= \frac{r^2}{r^2 + L^2} = 1 - \frac{L^2}{r^2 + L^2} \\
\highlight{ \rho(r) } &= \frac{M}{2\pi r} \frac{L^2}{(r^2 + L^2)^2} \\
\highlight{ m(r) } &= \frac{Mr^2}{L^2 + r^2} = M - \frac{LM}{L^2 + r^2}
\end{align}
\end{multicols}
\rule{0.9\textwidth}{0.4pt}
\begin{multicols}{2}
\subsection{NS 06.11.2013} \label{NS2013}
\begin{description}
  \item[Title] Holographic screens in ultraviolet self-complete quantum gravity
  \item[Keywords] Holography
  \item[Source] Elsevier Preprint by Mail am 12.11.13
\end{description}

\vfill
\columnbreak

\begin{align}
\highlight{ \rho(r) } &= \frac{M}{4 \pi r^2} \delta(r) \\
\delta(r) &= \dd{}{r} \Theta(r) \\
\Theta(r) &\to h(r) \\
\highlight{ \rho(r) } &= \frac{M}{4 \pi r^2} \dd{}{r} h(r) = T^0_0 \\
h(r) &= 1 - L^2 / (r^2 + L^2) \\
\sigma_h &= M / (4 \pi r_h^2)
\end{align}
\end{multicols}
\clearpage
\rule{0.9\textwidth}{0.4pt}
\begin{multicols}{2}
\subsection{NIM 07.11.2013}
\begin{description}
  \item[Title] Self-Completeness and the Generalized Uncertainity Principle
  \item[Keywords] -
\end{description}
Ein neues veröffentlichtes Paper auf ArXiv, parallel zum Preprint. Erstmals hübsche Bilder. Herleitung von $f(r)$ aus Operator $\cal A$:

\vfill
\columnbreak
\begin{align}
f(r) = 1 - 2 \frac{G M}{c^2 r} \gamma( 2; \frac{r}{\sqrt{\beta}} )
\end{align}
\end{multicols}
\rule{0.9\textwidth}{0.4pt}
\begin{multicols}{2}
\subsection{Rizzo 2006}\label{Rizzo2006}
\begin{description}
  \item[Title] Noncommutative inspired black holes in extra dimensions
  \item[Basiert auf]  NSS 2006 NC Ansatz (Section \ref{NSS 2006})
\end{description}

\vfill
\columnbreak
\begin{align}
\highlight{ \rho_\theta(r) } &= \frac{M}{(4 \pi \theta)^{3/2}} e^{-r^2 / 4 \theta} \\
&\to \frac{M}{(4\pi\theta)^{(n+3)/2}} e^{-r^2 / 4\theta}
\end{align}

\end{multicols}
\rule{0.9\textwidth}{0.4pt}

\section{Extension von [\ref{NS2012} NS2012] analog zu [\ref{Rizzo2006} Rizzo 2006] }
In Paper [\ref{NS2013} NS2013], in 4D, it was like (using $\Sigma := (r^2+L^2)^2$)

\begin{equation}
\rho(r) = \frac{M}{A_2} h'(r)
\propto \frac{1}{r^2} \frac{r}{\Sigma}
\quad\Rightarrow\quad
\mu(r) = \int_0^r \d r r^2 \rho(r)
\quad\Rightarrow\quad
\mu(r) = \int_0^r \d r \frac{r}{\Sigma}
\propto \left[ \frac{1}{\Sigma} \right]_0^r
\end{equation}

In the holography picture, only the $A_{(n-2)}$-Sphere, which is the surface of an $V_{(n-1)}$ dimensional matter ball in $(n-1)$ spacial dimensions (+1 time dimension makes $n$ space-time dimensions) seems to enter $\rho(r)$. So in $n$ dim, combining [\ref{NS2013} NS2013] + [\ref{Rizzo2006} Rizzo 2006]:

\begin{equation}
\rho(r) = \highlight{ \frac{M}{A_{(n-2)}}} \dd{h(r)}{r}
\quad \text{Units:} \left[ \rho \right] = \frac{[M]}{[A_{n-2}]} \left[ \dd{}{r}\right] [h] = \frac{E}{L^{n-2}} \frac{1}{L} \cdot 1 = \frac{E}{L^3} = \frac{1}{L^4} = E^4
\end{equation}

Formulas to remember for the volume of an $n$-Ball and its corresponding $(n-1)$-Sphere:
%
\begin{equation}
V_n = r^n \frac{\pi^{n/2}}{\Gamma\left( \frac{n}{2} + 1\right)}
\quad\text{and}\quad
A_{(n-1)} = \dd{V_n}{r} = \frac{\pi^{n/2}}{\Gamma\left(\frac{n}{2}+1\right)} n r^{n-1}
= 2 \frac{\pi^{n/2}}{\Gamma\left(\frac{n}{2}\right)} r^{n-1}
\end{equation}
\begin{equation}
\begin{aligned}
\Gamma(x) &= (x-1)! \\
\Gamma(x+1) &= x \Gamma(x)
\end{aligned}\quad
\text{Prefactors:}\quad 
\begin{aligned}
V_n &=v_n r^n \\
A_n &= a_n r^n
\end{aligned} \quad \text{Recursion:}\quad
\begin{aligned}
v_0 &= 1, &v_{n+1} &= a_n / (n+1) \\
a_0 &= 2, &a_{n+1} &= 2\pi v_n
\end{aligned}
\end{equation}
%
Now we evaluate the $(n-1)$ dimensional integral measure in spherical coordinates $k_\mu = (k_0, \vec{k}) = (k_0, r, \phi, \theta_1, \dots, \theta_{n-3})$, integrating only the spacial components:
%
\begin{equation} \label{nrad}
\int \d^{(n-1)}r = \int_0^\infty \d r ~ \highlight{r^{n-2}} \underbrace{\int_0^{2\pi} \d \phi ~~  \prod_{j=1}^{n-3} \int_0^\pi \d \theta_j ~ \sin^j(\theta_j)}_{=a_{n-2}\text{, since }(n-2)-\text{Surface}}
= \highlight{ \frac{2 \pi^{(n-1)/2}}{\Gamma\left(\frac{n-1}{2}\right)} \int_0^r \d r ~r^{n-2} }
\end{equation}

The factor $r^{n-2}$ is given by $(n-2)$ angles in $n$ dimensions.

The derivation of eqn. \ref{nrad} follows Wagner QFT2, not important here:
\begin{align}
B(x,y) &= \int_0^1 \d t t^{x-1} (1-t)^{y-1} = \frac{\Gamma(x)\Gamma(y)}{\Gamma(x+y)} \\
\Gamma(x) &= \int_0^\infty \d t ~ t^{x-1} e^{-t} \quad \text{und} \quad \gamma(s,x) = \int_0^x t^{s-1} e^{t} \d t, \quad \Gamma(\nicefrac{1}{2}) = \sqrt{\pi} \\
\quad \int_0^\pi \theta_j &~ \sin^j(\theta_j) = \frac{\sqrt{\pi}~ \Gamma\left(\frac{j+1}{2}\right)}{\Gamma\left(\frac{j+2}{2}\right)}
\end{align}

So in the end, it's straightforward for general $h(r)$:

\begin{equation}
\highlight{ \mu(r) } = a_{n-2} \int_0^r \d r r^{n-2} \rho(r) = a_{n-2} \int_0^r \frac{M}{A_{n-2}} h'(r) r^{n-2} \d r = \highlight{ M \int_0^r \d r \dd{h(r)}{r} } = M \left[ h(r) - h(0) \right]
\end{equation}

By construction of $\rho(r)$, it just kills the $(n-2)$ dimensional sphere. $\mu(r)$ only diverges if $h(0)$ diverges. The $h(r)$ Ansatz from [\ref{NS2012} NS2012] yields

\begin{equation}
h(r) = \frac{r^2}{r^2 + L^2}, \quad \dd{h(r)}{r} = \frac{2rL^2}{(r^2 + L^2)^2} \quad
\Rightarrow \quad \mu(r) = \frac{2r ~ML^2}{(r^2+L^2)^2}
\end{equation}

for \highlight{ any } dimension $n$.







% Literaturangaben
%\renewcommand{\refname}{Quellen}
%\begin{thebibliography}{99}
%\bibitem{Buch} W. Buckel, R. Kleiner: {\it Supraleitung}, Wiley-VHC
%\end{thebibliography}
\end{document}